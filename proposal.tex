\section{Proposal}

We propose a new standard attribute to introduce an attribute namespace
into the global scope.

\begin{lstlisting}
// Current situation
void f() {
  [[rpr::kernel, rpr::target(cpu,gpu)]]
  do_task();
}

// Proposed change
void g() {
  [[using(rpr), kernel, target(cpu,gpu)]]
  do_task();
}
\end{lstlisting}

In this paper we have named the proposed attribute \cppkey{using}. However,
this choice is for exposition only and a different name could be used.

\subsection{Effect of the using attribute}

The effect of a \cppid{using} attribute is to introduce all the attributes
names from a specific attribute namespace to the global attribute namespace.
Thus, after a \cppid{using} attribute, all the attributes from that namespace
can used without explicit mention to the namespace.

\begin{lstlisting}
void g() {
  [[using(rpr), kernel]] // equivalent to [[rpr::kernel]]
  do_task();
\end{lstlisting}

As a generalization, it should be possible to introduce multiple attribute
namespaces in a single attribute specifier.

\begin{lstlisting}
void g() {
  [[using(rpr,optimization), kernel, unroll(4)]]
  for (int i=0; i<max; ++i) {
    do_task(i);
  }
}
\end{lstlisting}

\subsection{Name lookup}

The introduction of the \cppkey{using} attribute requires additional
rules when an attribute specifier is seen. The question that needs to
be answered is how an attribute name is resolved.

If an attribute specifier contains a non-scoped attribute name, it is
first checked against the list of standard attributes. If it is found,
the attribute lookup is finished.

If a non-scoped attribute is a non-standard attribute, it is checked
against the attribute name lists of all the attribute namespaces that have been
introduced through an \cppkey{using} attribute. The outcome of this process
could be one of the following:

\begin{enumerate}

\item The lookup produces exactly one match. The program is considered
to be well formed.

\item The lookup produces more than one match in different attribute namespaces.
The program is considered to be ill-formed.

\item The lookup produces zero matches. The attribute is ignored and the
program is well-formed, although a diagnostic may be optionally emitted.

\end{enumerate}

\subsection{Scope of the using attribute}

There are several design alternative on the effect of a \cppid{using}
attribute in terms of scoping:

\begin{itemize}

\item Effect within the current attribute specifier.
\item Effect until the end of the current translation unit.
\item Effect within the current scope.

\end{itemize}

\subsubsection{Sequence specifier scope}

The simplest implementation is to make effect of a \cppid{using} attribute
limited to the current sequence specifier.

\begin{lstlisting}
void f(X & x) {
  [[rpr::kernel, rpr::target(gpu), rpr::out(x)]] g1(x); // OK
  [[using(rpr), kernel, target(gpu), out(x)]] g2(x); // OK
  [[using(rpr), kernel]] [[target(gpu)]] g3(x); // Wrong. Target in different attr-specifier
}
\end{lstlisting}

This option is simple to implement and provides some usefulness to the original
problem. For illustration we reproduce here our original pipeline example.

\begin{lstlisting}
void f() {
  [[using(rpr), pipeline(bound, 8, blocking), stream(A,B)]]
  for (int i=0; i<iterations; ++i) {
    [[using(rpr), kernel, out(a), target(cpu) ]]
    a = get_value();
    
    [[using(rpr), farm(4,ordered), in(A,C), out(A,B), target(cpu,gpu)]]
    for (int j=0;j<max;++j) {
      b = f(a,c);
    }

    [[using(rpr), kernel, in(A,B)]]
    g(a,b);
  }
}
\end{lstlisting}

This option makes easier to use multiple attributes from the same attribute
namespace. However, we still need to start with \cppkey{using} every attribute
specifier.

\subsubsection{Translation unit scope}

A second option, is that a \cppkey{using} attribute has effect from the point
in the source code where it is located to the end of the current translation unit.

This would allow to state a \cppkey{using} specifier only once.

\begin{lstlisting}
void f() {
  [[using(rpr)]]
  [[kernel,target(cpu)]] do_task1(); //rpr::kernel, rpr::target
  [[kernel,target(gpu),out(a)]] // rpr::kernel, rpr::target, rpr::out 
  for (int i=0; i<max; ++i) {
    a[i]=do_task2();
  }
}
\end{lstlisting}

However, there are several cases where the user could be easily surprised.

\begin{lstlisting}
void f() {
  [[using(rpr), kernel, target(cpu)]] // rpr::kernel, rpr::target
  do_task1();
}

void g() {
  [[kernel]]
  do_task3(); //rpr::kernel?
}
\end{lstlisting}

This situation could be made much worse by inserting \cppkey{using} attributes
in header files (e.g. inside an \cppkey{inline} function). Consequently, we
do not recommend this second alternative.

\subsubsection{Current scope}

A third option would be to use regular scoping rules for C++ lookup. That is
a \cppkey{using} attribute has effect from its point of use until the end of
the current scope.

\begin{lstlisting}
void f() {
  [[using(rpr), kernel, target(cpu)]] // rpr::kernel, rpr::target
  do_task1();
}

void g() {
  [[kernel]]
  do_task3(); // rpr::kernel not found!
}
\end{lstlisting}

We think this is the mos advisable solution. However, we recognize that it introduces
additional complexities as it makes necessary to mix regular namespace management
with attribute namespace management. Specific cases where complexities could arise
are attributes in namespace scope or even in the global scope.

One limited version of this approach could be to allow scope based \cppkey{using}
attributes at function body scope.

However for the sake of simplicity, we recommend the first option (sequence specifier scope).

\subsection{Why a standard attribute?}

It could be argued that if an extension needs an attribute as \cppkey{using} it could add
as part of the extension.

\begin{lstlisting}
void f() {
  [[rpr::using, kernel, target(cpu)]] // rpr::kernel, rpr::cpu
  do_task1();
}
\end{lstlisting}

However, this would be a non conforming extension as an implementation could see those
unknown attributes in the global namespace.

Besides, we think that the proposed \cppkey{using} attribute could be of utility to
multiple extensions.
